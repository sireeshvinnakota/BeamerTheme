\documentclass[12pt]{beamer}

%Input the Logic File
\usepackage{beamerThemeLogic}
% Specify Theme
\usetheme{UniMath}


 %\setbeamertemplate{footline}[frame number]{} % Uncomment this line if you want to remove the footer from each slide (and replace it with just the slide number (X/Y) in the bottom right of each slide.

%===============================================================%
% 				BEGIN YOUR PRESENTATION HERE					%
%===============================================================%

% Title and author information

\title[Short title]{Your Presentation Title}
% Remember to include both a short and a full title!
% The short title appears at the bottom of each slide.
\author{Your Name}
\institute[UST]{University of St. Thomas}
\date{\today}


%  \usepackage[sfmath]{kpfonts}
%  \renewcommand*\familydefault{\sfdefault}

%\setbeamerfont{frametitle}{shape=\scshape}

%===============================================================%
\begin{document}
%===============================================================%

\maketitle

\begin{frame}{Table of Contents (Summary)}
\tableofcontents
\end{frame}

\begin{frame}{Environments}
    \begin{theorem}
    hello
    \end{theorem}
    \begin{lemma}
    hey
    \end{lemma}
    \begin{prop}[Name of Proposition]
    contents
    \end{prop}
    \begin{cor}
    contents
    \end{cor}
\end{frame}

%===============================================================%
\section{\LaTeX Essentials}
%===============================================================%


\begin{frame}{More Environments}
	\begin{pro}
	Here's a hard problem
	\end{pro}
	\begin{defi}
	I'll define some stuff
	\end{defi}
	\begin{res}
	Result!
	\end{res}
	\begin{cl}
	Here's a claim I made
	\end{cl}
	\begin{mq}
	asdfk
	\end{mq}
\end{frame}




\begin{frame}{Itemized List}

	Itemized lists are punctuated by little shields (I am working on changing this!)
	\begin{itemize}
		\item Item
			\begin{itemize}
				\item Sub-item
			\end{itemize}
	\end{itemize}
	\begin{enumerate}%[label= (OPTION)]
	% options for enumerate: \alph*, \Alph*, \roman*, \arabic* (add punctuation also)
	\item
	\end{enumerate}
	\begin{description}
	\item[list item:] discussion
	\end{description}
\end{frame}
\begin{frame}{Columns}
\begin{columns}
	\begin{column}{0.5\textwidth}
   		some text here
	\end{column}
	\begin{column}{0.5\textwidth}
   		more text here
	\end{column}
\end{columns}
\end{frame}

\begin{frame}{Blocks}

	\begin{block}{Regular Block}
		Text goes here
	\end{block}

	\begin{alertblock}{Alert Block}
		Stands out a bit more
	\end{alertblock}

	\begin{exampleblock}{Example Block}
		Also stands out $y=\beta x+ \varepsilon$
	\end{exampleblock}

\end{frame}

%===============================================================%
\section{Animations}
%===============================================================%

\begin{frame}{On-Slide}

	Sometimes you want to hide later text/elements of a particular slide to keep the focus on the early part of the slide.

	\bigskip

	\onslide<2>{By having the text shaded out (and not completely missing), your audience can see that you do have some more information that will come shortly.}

\end{frame}
\begin{frame}{Pauses}
\begin{enumerate}
\item Item ONE
\pause
\item ITEM TWO!! 
\end{enumerate}
\end{frame}


%===============================================================%
\appendix
%===============================================================%
%Frames after this do not count towards the count.

\section{Thank You!}
{\BackgroundShaded
\begin{frame}
% BLANK FRAME AT THE END, can add a thank you!
\end{frame}
}

%===============================================================%
%===============================================================%





%===============================================================%
\section{First appendix section}
%===============================================================%

\begin{frame}{Appendix sample}

	Note that this slide doesn't count towards the total slides shown in the regular presentation

\end{frame}

\begin{frame}{hello}
    
\end{frame}

%===============================================================%
\end{document}
%===============================================================% 